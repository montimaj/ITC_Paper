\documentclass[review]{elsarticle}

\usepackage{lineno,hyperref, graphicx}
\usepackage[labelformat=simple]{subcaption}
\renewcommand\thesubfigure{(\alph{subfigure})}
\modulolinenumbers[5]

\journal{Int J Appl Earth Obs Geoinformation}

%%%%%%%%%%%%%%%%%%%%%%%
%% Elsevier bibliography styles
%%%%%%%%%%%%%%%%%%%%%%%
%% To change the style, put a % in front of the second line of the current style and
%% remove the % from the second line of the style you would like to use.
%%%%%%%%%%%%%%%%%%%%%%%

%% Numbered
%\bibliographystyle{model1-num-names}

%% Numbered without titles
%\bibliographystyle{model1a-num-names}

%% Harvard
%\bibliographystyle{model2-names.bst}\biboptions{authoryear}

%% Vancouver numbered
%\usepackage{numcompress}\bibliographystyle{model3-num-names}

%% Vancouver name/year
%\usepackage{numcompress}\bibliographystyle{model4-names}\biboptions{authoryear}

%% APA style
\bibliographystyle{model5-names}\biboptions{authoryear}

%% AMA style
%\usepackage{numcompress}\bibliographystyle{model6-num-names}

%% `Elsevier LaTeX' style
%\bibliographystyle{elsarticle-num}
%%%%%%%%%%%%%%%%%%%%%%%

\tolerance=1
\emergencystretch=\maxdimen
\hyphenpenalty=10000
\hbadness=10000
\sloppy

\begin{document}

\begin{frontmatter}

\title{Pol-InSAR and PolSAR based Inversion Modelling for Snow Depth and SWE Estimation in the Northwestern Himalayan Watershed}
%\tnotetext[mytitlenote]{Fully documented templates are available in the elsarticle package on \href{http://www.ctan.org/tex-archive/macros/latex/contrib/elsarticle}{CTAN}.}

%% Group authors per affiliation:
%\author{Sayantan Majumdar\fnref{myfootnote}, Praveen K. Thakur, Ling Chang, Shashi Kumar}
%\address{Faculty ITC, University of Twente}
%\fntext[myfootnote]{Since 1880.}

%% or include affiliations in footnotes:
\author[itc,iirs]{Sayantan Majumdar \corref{corrauth}}
\cortext[corrauth]{Corresponding author}
\ead{monti.majumdar@gmail.com}

\author[iirs]{Praveen K. Thakur}
%\ead{praveen@iirs.gov.in}
\author[itc]{Ling Chang}
%\ead{ling.chang@utwente.nl}
\author[iirs]{Shashi Kumar}
%\ead{shashi@iirs.gov.in}

\address[itc]{Faculty of Geo-information Science and Earth Observation (ITC), University of Twente}
\address[iirs]{Indian Institute of Remote Sensing (IIRS), ISRO}

\begin{abstract}
Snow depth (SD) and Snow Water Equivalent (SWE) constitute essential physical properties of snow and find extensive usage in the hydrological modelling domain. However, the prominent influence of the hydrometeorological conditions present in the area of interest inhibits accurate large-scale measurement of the SD and SWE--- an ongoing research problem in the cryosphere paradigm. In the past few decades, synthetic aperture radar (SAR) has been widely used in the cryospheric studies which mainly concern with the snow property retrieval, such as SD, SWE, and snow density. Moreover, spaceborne SAR systems benefit from global coverage at sufficiently high spatial resolutions. Previously, the copolar phase difference (CPD) method based on the X-band polarimetric SAR (PolSAR) technique has displayed promising results regarding the fresh snow depth (FSD) estimation. Still, this FSD inversion model has not been tested in the presence of extreme topographically varying conditions, such as the northwestern Himalayan belt. It is also susceptible to high volume scattering at X-band occurring from the increased snow grain sizes as a result of the standing (or old) snow formation driven by the temperature induced snow metamorphosis process. Hence, to model this volume decorrelation, the polarimetric SAR interferometry (Pol-InSAR) technique can be applied which has already provided highly accurate tree height estimates in prior studies. In this work, the FSD and standing snow depth (SSD) are computed using the PolSAR CPD method and the single-baseline Pol-InSAR based hybrid Digital Elevation Model (DEM) differencing and coherence amplitude inversion model. To achieve this, the TerraSAR-X, TanDEM-X Coregistered Single look Slant range Complex (CoSSC) bistatic acquisition over Dhundi (situated in the Beas watershed, northwestern Himalayas, India) on January 8, 2016, is used. Although meant for flexibility, these models involve several free parameters requiring data specific optimisation. Moreover, since the study area is characterised by steep slopes and forests, there exist significant uncertainty sources which exhibit temporally varying scattering mechanisms. Additionally, the ground-truth measurements are limited (only two points are available, with one falling in the layover area for descending pass acquisitions). As a result, appropriate sensitivity analyses have been carried out for the parameter optimisation. Furthermore, the uncertainty sources are identified by performing a summer (June 8, 2017) and wintertime (January 8, 2016) comparative analysis of the study area which quantitatively highlights the changes in the percentages of the surface and volume scatterings. Apart from this, the DEM error analysis using the elevation readings acquired during the fieldwork showed that the elevation errors do not significantly modify the local incidence angle values which are involved in the FSD and SSD inversion algorithms. Evidently, the improved models display sufficiently high FSD and SSD accuracies of 94.83\% and 99.53\% respectively with the corresponding fresh SWE (FSWE) and standing SWE (SSWE) accuracies of 94.84\% and 99.48\% (measured using a 3$\times$3 window at Dhundi). Therefore, in summary, the overall outcome of this research showcases the practicability of these PolSAR and Pol-InSAR models in the context of the SD estimation over rugged terrains.
\end{abstract}

\begin{keyword}
Synthetic Aperture Radar, Copolar Phase Difference, Pol-InSAR, Snow Physical Properties, Sensitivity Analysis
\end{keyword}

\end{frontmatter}

\linenumbers

\section{Introduction}
\label{sec:Intro}
The cryosphere collectively represents the regions of the Earth where water is prevalent in its solid form, either permanently (annually) or temporarily (seasonally). These include the polar ice caps, and the snow covered mountainous areas, all of which significantly contribute to the global climate system change. Evidently, snow is the second most extensive component of the cryosphere after frozen ground having maximum and mean cover extents of approximately 47 million sq. km (in January) and 26 million sq. km respectively \citep{Barry2011}. As a result, the frequent large-scale monitoring of snow is central to implementing environmental policies, for which remote sensing is the only way forward \citep{Tedesco2015}.

Snow depth and snow water equivalent constitute two of the most important physical properties of snow and are extensively used in hydrological models that relate to snowmelt runoff and snow avalanche predictions \citep{Thakur2017}. While snow depth or snow height refers to the distance of the ground to the snow surface, SWE quantifies the amount of water present in a snowpack (layered snow formed by accumulation over time). Theoretically, SWE is defined as the product of snow depth and snow density and can be conceptualised as the amount of liquid water obtained owing to the instantaneous melting of
an entire snowpack \citep{Tedesco2015}. The accurate estimation of these two parameters is quite challenging depending upon the data availability and variety, mathematical model selection, and the hydrometeorological conditions of the area of interest. Hence, it is considered to be an important research element in the cryosphere paradigm \citep{Leinss2014, Leinss2015}.

Due to the difficulties posed by in-situ or ground based measurements of snow depth and SWE in rugged terrains, remote sensing techniques coupled with adequately sampled (both in space and time domains) ground measurements are widely used to improve the quality of these estimated parameters over considerably large areas \citep{Takala2011}. Currently, LiDAR (Light Detection and Ranging) and spaceborne SAR are the most popular techniques used in the studies related to snow, ice and the cryosphere in general \citep{Deems2013, Leinss2014, Tedesco2015}. However, LiDAR can only be used to determine the height of the snow and cannot be used for measuring other physical properties such as snow density and snow wetness. In addition, the operating cost of LiDAR is sufficiently high and is also weather dependent \citep{Deems2013}. As a result, spaceborne SAR systems benefit from substantial coverage (globally available), cloud insensitivity, night-time operability and are extensively used to measure the snow physical properties sufficiently at high spatial resolutions \citep{Moreira2013, Thakur2012}.

The applicability of SAR systems for snow cover monitoring was discussed as early as 1977 \citep{Ulaby1977} wherein the snow backscatter coefficient was measured and was thereafter modelled for various frequencies, layers, and polarisations \citep{Zuniga1979}. It was shown that only very high microwave frequencies (Ku-band or higher) exhibit a significant dependence on SD or the SWE of dry or standing (deposited) snow \citep{Yueh2009}. However, lower frequencies (X-
band or below) penetrate through dry snow whereby the underneath frozen soil or ground primarily contributes to the radar backscatter signal. On the other hand, in case of moist snow (the transitional stage between dry and wet snow) and wet snow, the predominant scattering occurs from the snow volume and snow surface respectively due to the presence of water. Essentially, water, with its high dielectric constant, heavily modifies the dielectric properties of snow and effectively reduces the snow penetration capacity of the radar pulses \citep{Abe1990}. The radar backscattering mechanism for a typical snow covered area can be conceptualised from Figure \ref{fig:concept}. In principle, PolSAR and InSAR systems utilise these received target echoes for supporting various microwave remote sensing applications in the cryosphere domain.

\begin{figure}[htb]
    \centering
    \includegraphics[width=\textwidth]{Figures/Conceptual.png}
    \caption{Conceptual diagram displaying the radar backscattering mechanism in hilly terrains. Adapted from \cite{Thakur2012}.}
    \label{fig:concept}
\end{figure}

A polarimetric SAR system utilises the polarised radar echoes to obtain information about the specific scattering mechanism for a particular target. In essence, by using a coherent analysis which incorporates the phase of different polarimetric channels, it is possible to differentiate various scattering mechanisms \citep{Lee2009}. Nowadays, PolSAR based algorithms that work on the polarimetric backscatter signal have been widely adopted for various snow related applications such as the classification of dry and wet snow, measuring snow wetness and snow density \citep{Singh2017, Snehmani2010, Thakur2012, Thakur2017, Usami2016}. In this context, the roll-invariant entropy-anisotropy-alpha (H-A-$\alpha$) decomposition and Wishart classification have been successfully tested to classify different snow types as well as demarcate snow covered areas \citep{Cloude2010,Lee2009,Singh2014}. A few years back, the use of spaceborne PolSAR for snow height determination had been introduced, wherein the relationship between the copolar phase difference (CPD) and fresh snow depth (FSD) is quantitatively analysed by deriving a theoretical model \citep{Leinss2014}. However, the major challenge in this approach lies in accurately modelling the anisotropic effective permittivity of dry snow which is dependent on the depolarisation factor (calculated by fixing the shape of the ice grain) and ice grains’ volume fraction (measured using snow density).

Interferometric SAR techniques find significant usage in the cryosphere domain and have been used to construct highly accurate digital elevation models (DEMs), measure dry snow depth and SWE in several studies \citep{Guneriussen2001, Lei2016, Leinss2015, Li2017, Liu2017}. The principle of SAR interferometry builds upon measured phase differences between radar images of the same area acquired at different temporal instances (repeat-pass) or different viewing geometries but same epoch (single-pass) \citep{Hanssen2001}. Still, the inherent problems of spatial and temporal decorrelations and atmospheric inhomogeneities are the primary limiting factors in the studies involving InSAR and its variant D-InSAR (Differential InSAR) \citep{Pepe2017}. While spatial decorrelation is caused by large perpendicular baselines \citep{Pepe2017}, the problem of temporal decorrelation arises due to the change in the surface over time \citep{Leinss2018, Leinss2015}. Moreover, the atmospheric noise occurs owing to the variation in the water vapour distribution in the atmosphere \citep{Hanssen2001}. These factors are responsible for inaccurate and low-coherence measurements, thereby leading to a potential decrease in the accuracy of the final results. In cryosphere research, the loss of coherence in InSAR is heavily influenced by the snow humidity, melting, and refreezing and is also susceptible to the variations in both spatial and temporal baselines. Although data assimilation algorithms like 3DVAR (three dimensional variation) and EnKF (Ensemble Kalman Filter) have been applied to the produced outputs of the SD inversion models for minimizing the effect of temporal decorrelation, the applicability and feasibility of such algorithms remains untested on varying data sets and study areas \citep{Liu2017}.

The Pol-InSAR technique works on the coherent combination of both PolSAR and InSAR observations, thereby enabling the interferogram generation in arbitrary transmit and receive channels \citep{Cloude1998, Cloude2005, Cloude2010}. It has been widely used for estimating tree height in forested regions and can be effectively applied to natural or artificial volume scatterers including snow and ice \citep{Hajnsek2009, Kugler2015, Kumar2017, Papathanassiou2001}. In essence, the identification of different scattering processes (PolSAR) and the vertical profile sensitivity (InSAR) are unique to this technique. Therefore, the applicability of Pol-InSAR based SD retrieval could prove its potential in case of the standing snow depth (SSD) \citep{Negi2009, Thakur2012, Thakur2017}.

The prime focus of this research is to estimate the FSD and SSD using PolSAR and Pol-InSAR respectively. In addition, the corresponding fresh SWE (FSWE) and standing SWE (SSWE) are to be determined, for which the respective snow densities need to be known. Essentially, the study involves sensitivity analysis of the various model parameters along with the identification of the different uncertainty sources present in the chosen geographical area. 

This manuscript is compartmentalised into seven sections each consisting of several subsections. It starts with an introductory discussion in section \ref{sec:Intro}. Thereafter, the study area and the datasets including the required  software are specified in section \ref{sec:study}. From section \ref{sec:method} onwards the methodology and results are discussed. Finally, the relevant conclusions and recommendations are put forward in section \ref{sec:conc}.

\section{Study Area, Datasets, and Software}
\label{sec:study}

\subsection{Chosen Study Area}
\subsubsection{Geographical Situation}
The Beas river watershed near Manali, India is part of the north-western Himalayas. Naturally, steep slopes and dense forests are prominent in this region. The elevation typically varies from nearly 2500 m to more than 5000 m in some places as observed in the reference ALOS PALSAR DEM (Figure \ref{fig:overview}). In this work, a small region ($\sim$ 96 km\textsuperscript{2}) of the Beas basin is chosen which starts a few kilometres uphill from Dhundi up to Kothi (shown in Figure \ref{fig:overview}). These areas receive substantial seasonal snowfall which begins in December and lasts till late March. However, the cold, dry season usually commences from late September or early October. The coldest period is in January during which the temperatures reach a daily minimum of -15$^\circ$C on an average. The summers are mild to occasionally warm with June being the hottest month (mean and maximum temperatures of 20$^\circ$C and 30$^\circ$C respectively are common). Apart from this, significant rainfall occurs between late June and September (monsoon season) with August receiving the maximum precipitation \citep{Majumdar2019, Thakur2012}.

\begin{figure}[htb]
    \centering
    \includegraphics[width=\textwidth]{Figures/Overview.png}
    \caption{Overview map of the study area showing the ALOS PALSAR DEM. The original DEM of 12.5 m spatial resolution (generated in 2011) has been resampled to 3 m using bilinear interpolation \citep{Wu2008} to match the high resolution SAR data. Moreover, the vertical resolution as per the product specification is 5 m.}
    \label{fig:overview}
\end{figure}

\subsubsection{Field Visit}
Intensive fieldwork had been conducted from October 14-21, 2018 in the Dhundi and Kothi areas where several Differential Global Positioning System (DGPS) measurements were acquired using the Leica Viva GS 10 \citep{LeicaGeosystemsAG2012} with adequate horizontal positional accuracies ($\sim$7 cm) \citep{Majumdar2019}. Due to the complex terrains, most of the DGPS readings had been obtained through the kinematic mode \citep{Luo2014}. However, in some of the convenient places such as the Dhundi base station and near the Kothi Automatic Weather Station (AWS), the static mode was used \citep{LeicaGeosystemsAG2012}. Eventually, elevation information from these DGPS points have been compared with the ALOS PALSAR DEM, the details of which are provided in section \ref{sec:res}. Furthermore, the manual snow readings from 2014-2018 (snow depth, density, weather profile and other relevant data) which are maintained by the security personnel daily at Dhundi had been pagewise photographed using a smartphone camera. In order to properly understand and visualise the characteristics of the study area, selected field photographs and their brief description are shown from figures \ref{subfig:gps}-\ref{subfig:stations}.

\begin{figure}[htb]
    \centering
    \begin{subfigure}[t]{0.49\textwidth}
        \raisebox{-\height}{\includegraphics[width=\textwidth]{Figures/Field/me_gps.jpg}}
        \caption{DGPS positional accuracy checking}
        \label{subfig:gps}
    \end{subfigure}
    \hfill
    \begin{subfigure}[t]{0.49\textwidth}
        \raisebox{-\height}{\includegraphics[width=\textwidth]{Figures/Field/base.png}}
        \caption{Leica DGPS base}
        \label{subfig:base}
    \end{subfigure}
    %%%%%%%%%%%%%%%%%%%%%%%%%%%%%%%%%%%%second row
    \begin{subfigure}[t]{0.49\textwidth}
        \raisebox{-\height}{\includegraphics[width=\textwidth]{Figures/Field/beas.jpg}}
        \caption{Beas river}
        \label{subfig:beas}
    \end{subfigure}
    \hfill
    \begin{subfigure}[t]{0.49\textwidth}
        \raisebox{-\height}{\includegraphics[width=\textwidth]{Figures/Field/landscape.jpg}}
        \caption{Landscape and human settlements}
        \label{subfig:landscape}
    \end{subfigure}
    %%%%%%%%%%%%%%%%%%%%%%%%%%%%%%%%%%%third row
    \begin{subfigure}[t]{0.49\textwidth}
        \raisebox{-\height}{\includegraphics[width=\textwidth]{Figures/Field/mountain.jpeg}}
        \caption{Mountains and forests}
        \label{subfig:mountain}
    \end{subfigure}
    \hfill
    \begin{subfigure}[t]{0.49\textwidth}
        \raisebox{-\height}{\includegraphics[width=\textwidth]{Figures/Field/stations.jpg}}
        \caption{Weather instruments}
        \label{subfig:stations}
    \end{subfigure}
    \caption{Dhundi field photographs showing the varying topographic features present in the surrounding area.}
    \label{fig:field}
\end{figure}

\subsection{Datasets Used}

Overall twelve Coregistered Single look Slant range Complex (CoSSC) TerraSAR-X (TSX)/TanDEM-X (TDX) bistatic X-band SAR images acquired between December 2015 and August 2017 in stripmap (SM) mode are available over this study area \citep{Balss2012}. In total, there are six Quad-pol data pairs, from which the descending orbital pass acquisition at 00:53 hrs Universal Time Coordinated (UTC), January 8, 2016, has been selected considering the occurrence of fresh snowfall before, during and after the satellite flyby. Moreover, the perpendicular baseline ($B_\bot$) and ambiguity height ($h_{2\pi}$) for this data are 96.34 m and 63.18 m respectively. 

Additionally, the in-situ snow physical parameters’ data (standing and fresh snow depths, snow density) along with the relevant weather data had been transferred to a PostgreSQL database (DB) \citep{PostgreSQL2019} from the photographs of the manual recordings through spreadsheets. Apart from this, the high frequency data (two-minute interval measurements) obtained from the snowpack analyser (SPA) device (installed at Dhundi) had been downloaded and were added to the database as a separate table. Accordingly, the SSDs at 06:22 hrs (00:52 hrs UTC) Indian Standard Time (IST) on January 7, 2016, and 06:22 hrs January 8, 2016 morning were 36.2 cm and 54.9 cm respectively signifying a heavy fresh snowfall event of 18.7 cm within 24 hrs. The manual recordings also showed an FSD of 18 cm on January 8, 2016 morning though the exact measurement time is unspecified in the record book. Apart from this, a forest mask used in the earlier studies of this area \citep{Thakur2012, Thakur2017} has been obtained from the Water
Resources Department (WRD), Indian Institute of Remote Sensing (IIRS).

\subsection{Software}
The Sentinel Application Platform (SNAP) 6.0.5 \citep{ESA2018} has been used for basic SAR processing. In addition, the FSD and SSD inversion models have been implemented using Python 3 wherein PyCharm Community Edition 2018.1 \citep{JetBrains2018} was used as the Integrated Development Environment (IDE). Moreover, the final SD and SWE maps have been prepared using QGIS 2.18 \citep{QGIS2016}. Furthermore, some of the computationally intensive tasks have been carried out using the High-Performance Computing (HPC) infrastructure installed at IIRS.

\section{Methodology}
\label{sec:method}
There are various bibliography styles available. You can select the style of your choice in the preamble of this document. These styles are Elsevier styles based on standard styles like Harvard and Vancouver. Please use Bib\TeX\ to generate your bibliography and include DOIs whenever available.

Here are two sample references: \cite{Thakur2012, Leinss2014, Cloude1998, Cloude2010}.

\section{Results and Discussion}
\label{sec:res}
\section{Conclusion}
\label{sec:conc}

\section*{References}

\bibliography{refs}

\end{document}